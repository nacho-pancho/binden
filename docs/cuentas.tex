\documentclass{report}
\usepackage[utf8]{inputenc}
\usepackage{amsmath,amssymb,amsfonts}
\def\opt{\ensuremath{^{*}}}
\def\loss{\ell}
\def\expect{\ensuremath{\mathbb{E}}}
\title{Cuentas DUDE Poisson-Exp}
\author{Ignacio Ramirez}
\date{\today}
\begin{document}
\maketitle

Formula de dude para $\ell_2$. 

\begin{align}
x\opt(c,z) &= \expect_{q(X|C=c,Z=z)}\left[\;x\;\right]
\end{align}
con

\begin{align}
q(X=x|c,z) &=
\frac{p(Z|X=x)p(X=x|C=c)}{\int_0^{\infty}{p(Z|X=x)p(X=x|C=c)}} \\
 &= \frac{1}{\kappa(c,z)p(Z|X=x)p(X=x|C=c)} p(Z|X=x)p(X=x|C=c) 
\end{align}

        %% #
        %% # given that the Channel transition distribution P(Z|X) is Poisson with parameter theta 
        %% # if we assume that the input distribution P(X) is an Exponential with parameter alpha, P(X|alpha) = alpha exp(-alpha*x)
        %% # then the observed output (noisy) distribution P(Z) will be Exponential with parameter p = alpha/(alpha+1)
        %% # with mean (1-p)/p = 1/alpha
        %% # then, to invert the channel, we only need to compute the average, then alpha from the average
        %% # and we readily have P(X|c)!

Tengo
\begin{align}
p(X=x|C=c) &= \mu^{-1} e^{-x/\mu}\\
p(Z=z|X=x) &= \frac{x^z}{z!}e^{-x}
%\loss(x,z) &= (x-z)^2.
\end{align}

Para $\kappa(c)$ tengo,

\begin{align}
\kappa(c,z) &= 
  \int_{0}^{\infty}
  {
    \frac{1}{\mu}
    e^{-\zeta/\mu}
    \frac{\zeta^z}{z!}
    e^{-\zeta} 
    d\zeta
  } \\
\kappa(c,z) &= 
  \frac{1}{\mu{z!}}
  \int_{0}^{\infty}
  {
    \zeta^{z}
    e^{-\zeta/\mu}
    e^{-\zeta} 
    d\zeta
  } \\
\kappa(c,z) &=
  \frac{1}{\mu{z!}}
  \int_{0}^{\infty}
  {
    \zeta^{z}
    e^{-(1+1/\mu)\zeta}
    d\zeta
  } \\
\kappa(c,z) &=
\frac{1}{\mu{z!}}\int_{0}^{\infty}
{
  \left(
    \frac{\theta}{1+1/\mu}
  \right)^{z}
  e^{-\theta}
  \frac{1}{1+1/\mu}
  d\theta
}
\quad [cv:\;\theta=(1+1/\mu)\zeta] \\
\kappa(c,z) &=
\frac{1}{\mu(1+1/\mu)^{z+1}z!}
\int_{0}^{\infty}
{
  \theta^{z}
  e^{-\theta}
  d\theta
} \\
\kappa(c) &=
\frac{1}{\mu(1+1/\mu)^{z+1}z!}
\Gamma(z+2) = 
\frac{1}{\mu(1+1/\mu)^{z+1}z!} (z+1)!
\\
\kappa(c) &= \frac{1}{\mu(1+1/\mu)^{z+1}}
\end{align}

La formula del denoiser es casi idéntica, con un $\zeta$ ``de más'',

\begin{align}
x\opt(c,z) &= 
  \kappa(c)\int_{0}^{\infty}
  {
    \zeta 
    \frac{1}{\mu}
    e^{-\zeta/\mu}
    \frac{\zeta^z}{z!}
    e^{-\zeta} 
    d\zeta
  } \\
x\opt(c,z) &= 
  \frac{\kappa(c)}{\mu{z!}}
  \int_{0}^{\infty}
  {
    \zeta^{z+1}
    e^{-\zeta/\mu}
    e^{-\zeta} 
    d\zeta
  } \\
x\opt(c,z) &=
  \frac{\kappa(c)}{\mu{z!}}
  \int_{0}^{\infty}
  {
    \zeta^{z+1}
    e^{-(1+1/\mu)\zeta}
    d\zeta
  } \\
x\opt(c,z) &=
\frac{\kappa(c)}{\mu{z!}}\int_{0}^{\infty}
{
  \left(
    \frac{\theta}{1+1/\mu}
  \right)^{z+1}
  e^{-\theta}
  \frac{1}{1+1/\mu}
  d\theta
}
\quad [cv:\;\theta=(1+1/\mu)\zeta] \\
x\opt(c,z) &=
\frac{\kappa(c)}{\mu(1+1/\mu)^{z+2}z!}
\int_{0}^{\infty}
{
  \theta^{z+1}
  e^{-\theta}
  d\theta
} \\
x\opt(c,z) &=
\frac{1}{\mu(1+1/\mu)^{z+2}z!}
\Gamma(z+2) = 
\frac{\kappa(c)}{\mu(1+1/\mu)^{z+2}z!} (z+1)!
\\
x\opt(c,z) &= \frac{1}{\kappa(c)}\frac{z+1}{\mu(1+1/\mu)^{z+2}} \\
x\opt(c,z) &= \mu(1+1/\mu)^{z+1}\frac{z+1}{\mu(1+1/\mu)^{z+2}} \\
x\opt(c,z) &= \frac{z+1}{1+1/\mu} \\
\end{align}


\end{document}
